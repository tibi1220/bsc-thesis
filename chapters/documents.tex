%! TEX root = ../thesis.tex

% ------------------------------------------------------------------------------
\cleardoublepage
% ------------------------------------------------------------------------------
\chapter{Az oktatási anyagok}
% ------------------------------------------------------------------------------

% This is a small summary of this chapter
Ebben a fejezetben az elkészült oktatási anyagokat fogom bemutatni.
Egy rövid történelmi áttekintő után bemutatom, hogy milyen megfontolás alapján
döntöttem a \LaTeX{} nyelv használata mellett. Bemutatom a technológia előnyeit
a szokványos szövegszerkesztőkkel szemben, megmutatom egy ilyen dokumentum
általános felépítéseit, és a leggyakrabban használt gyári parancsokat.
Az alapok után kitérek arra, hogy hogyan lehet saját parancsokat, környezeteket,
csomagokat és osztályokat létrehozni, és hogyan lehet ezeket felhasználni,
ezzel is megkönnyítve a dokumentumok szerkesztését.

% ------------------------------------------------------------------------------
\section{A \LaTeX{} dokumentumok általános tulajdonságai}
% ------------------------------------------------------------------------------

% This is a summary of waht LaTeX is
A \LaTeX{} egy olyan dokumentumkészítő rendszer, melyet széles körben használnak
tudományos munkák készítésére, különösen olyan területeken, mint a matematika,
fizika, kémia, vagy éppen az informatika. A \LaTeX{} más dokumentumszerkesztő
szoftverekkel (pl. \textit{Microsoft Word}, \textit{LibreOffice Writer})
ellentétben nem a \textit{WYSIWYG} (What You See Is What You Get -- Amit látsz,
azt kapod) elvet követi. Az ilyen szövegszerkesztők -- a \LaTeX{}-kal
-- ellentétben egy olyan interaktív felületet biztosítanak, amelyen a
felhasználó a dokumentumot szerkesztés közben úgy látja, ahogyan az a nyomtatott
verzióban meg fog jelenni. A \LaTeX{} esetében ez a folyamat teljesen más. A
domunetum egy egyszerű szövegfájl (általában \texttt{.tex} kiterjesztéssel),
amelyben a szerző a szöveg mellett speciális parancsokat is használhat, melyek
meghatározzák a dokumentum felépítését, formáját, stílusát, és egyéb
tulajdonságait. Ezt a szövegfájlt egy \TeX motornak nevezett program fordítja
le PDF formátumú dokumentummá. A \LaTeX{} tehát egy leíró nyelv, amelyet a
szerzők a dokumentum tartalmának megadására használnak, és amelyet a
fordítóprogramok a dokumentum formázására használnak. \cite{overleaf_30}

% ------------------------------------------------------------------------------
\subsection{A \LaTeX{} története}
% ------------------------------------------------------------------------------

% A brief history of LaTeX
A \TeX{} eredetileg Donald Knuth matematikus professzor számítógépes
tipográfiai rendszerének neve volt. Knuth a hetvenes években kezdett el
foglalkozni a témával, mivel elégedetlen volt a korabeli számítógépes
nyomtatási technológiával. Célja az volt, hogy egy olyan rendszert hozzon létre,
amellyel szép könyvek készíthetőek, különösen olyanok, amelyekben sok
matematikai képlet is szerepel. \cite{texbook}

A \TeX{} abban az időben forradalmi újításnak számított, hiszen lehetővé tette
a dokumentumok felépítése, a szöveg megjelenése, és a matematikai egyenletek
feletti teljes kontrollt. Hamar népszerűvé vált a tudományos közösségben, hiszen
egyszerűen lehetett vele komplex dokumentumokat készíteni, melyek megjelenése
is minőségibb volt, mint a korabeli szövegszerkesztőkkel készített
dokumentumoké.

Az 1980-as években Leslie Lamport, a Digital Equipment Corporation kutatója
arra a felismerésre jutott, hogy bár a \TeX{} egy rendkívül hatékony eszköz, a
kezdők számára használata korántsem egyszerű. Ennek a feloldására Lamport
olyan makrókat készített, melyek felhasználóbarátabb interfészt biztosítottak a
dokumentumszerkesztésbe belevágó kalandorok számára. Ezt a rendszert
\LaTeX{}-nek nevezék el, amely a \textit{Lamport's \TeX} rövidítése.
\cite{latex2e}

% ------------------------------------------------------------------------------
\subsection{A \LaTeX{} előnyei}
% ------------------------------------------------------------------------------

% Why use LaTeX?
A \LaTeX{} rendszert gyakran a \textit{WYSIWYG} dukumentumszerkesztőkkel
szokás összehasonlítani. Nézzük meg, hogy miért érdemes tudományos munkák
készítéséhez a \LaTeX{}-et használni.

\begin{enumerate}
	\item \textbf{Rendkívül jól kezeli a komplex dokumentumokat}:
	      A \LaTeX{} kiváló választás összetett szerkezetű, nagy terjedelmű
	      dokumentumok -- mint például egy diplomamunka -- készítéséhez.
	      Biztosítja a dokumentumok könnyű szerkeszthetőségét és
	      konzisztenciáját.

	\item \textbf{A tartalom és a stilisztika kettéválasztása}:
	      A \LaTeX{} különválasztja a dokumentum tartalmát a stílusától. Ez
	      elősegíti azt, hogy a szerzőknek csak a tartalomra kelljen koncentrálni,
	      a dokumentum kiváló megjelenését a \LaTeX{} fogja biztosítani.

	\item \textbf{Fejlett matematikai képességek}:
	      A \LaTeX{} segítségével gyorsan és egyszerűen lehet matematikai
	      képleteket és szimbólumokat beilleszteni a dokumentumokba.

	\item \textbf{Testreszabhatóság}:
	      A \LaTeX{} rendelkezik olyan eszközökkel, amelyek lehetővé teszik a
	      felhasználók számára, hogy a dokumentumok megjelenését a saját
	      igényeikhez igazítsák.

	\item \textbf{Automatizált bibliográfiakezelés}:
	      A \LaTeX{} rendelkezik olyan eszközökkel, amelyek lehetővé teszik a
	      hivatkozások automatikus kezelését. A szerzőnek csak meg kell adnia a
	      hivatkozásokat tartalmazó adatbázist, a \LaTeX{} pedig ezek alapján
	      automatikusan generálja a hivatkozásokat.

	\item \textbf{Platformfüggetlenség}:
	      A \LaTeX{} egy olyan nyelv, amelyet bármilyen operációs rendszeren
	      lehet használni. A szerzőnek csak egy szövegszerkesztőre van szüksége.

	\item \textbf{Kollaboráció és verziókezelés}:
	      Mivel a \LaTeX{} dokumentumok forrásai egyszerű szövegfájlok, ezért
	      könnyen kezelhetőek verziókezelő rendszerekkel.

	\item \textbf{Nyílt forráskód}:
	      A \LaTeX{} egy nyílt forráskódú szoftver, amelyet bárki szabadon
	      felhasználhat, módosíthat, és terjeszthet. A \LaTeX{}-et a
	      \textit{\LaTeX{} Project} fejleszti. \cite{latex_project}
\end{enumerate}

% ------------------------------------------------------------------------------
\section{A \LaTeX{}-kel készített dokukenumok felépítése}
% ------------------------------------------------------------------------------

% The structure of a LaTeX document
Egy \LaTeX{} dokumentum két fő részből áll: a \textit{preambulumból} és a
\textit{dokumentum törzsből}. A preambulum a dokumentum elején található, és a
dokumentum formázásához szükséges beállításokat tartalmazza. A törzsben pedig
maga a tartalmai rész jelelentik meg. Az alábbi kódrészletben egy \LaTeX{}
dokumentum egy egyszerű példáját láthatjuk:

\begin{lstlisting}[caption={\LaTeX{} mintaprogram},language=tex]
  % Preambulum
  \documentclass{article}

  % Törzs
  \begin{document}
    Hello World!
  \end{document}
\end{lstlisting}


% ------------------------------------------------------------------------------
\subsection{A preambulum}
% ------------------------------------------------------------------------------

% The preamble
A preambulum a dokumentum elején található, a
\inlinecode{\textbackslash begin\{document\}}
parancs előtt. Ebben a részben definiálhatjuk a dokumentum osztályát, a használt
csomagokat, és a dokumentum egyéb tulajdonságait.

Oszályt a
\inlinecode{\textbackslash documentclass}
paranccsal adhatunk meg. Használhatunk a \textit{\LaTeX{} Project} által
biztosított osztályokat (pl. \texttt{article}, \texttt{report}, \texttt{book}),
harmadik fél által készített osztályokat (pl. \texttt{scrartcl},
\texttt{scrbook}), vagy akár saját osztályokat is. Ezek az osztályok a
dokumentumok különböző típusait reprezentálják. Az \texttt{article} osztály
például rövid dokumentumok készítésére alkalmas, míg a \texttt{book} osztály
hosszabb dokumentumok készítésére. Az osztályok definiálják a dokumentum
alapértelmezett formázását, valamint a dokumentumhoz tartozó parancsokat és
környezeteket. \cite{latex_project}

A dokumentum osztályának megadása után a
\inlinecode{\textbackslash usepackage}
paranccsal csomagokat tölthetünk be. A csomagok olyan kiegészítő modulok,
amelyek új parancsokat és környezeteket definiálnak, vagy a meglévőket
módosítják. Bemutatok néhány gyakran használt csomagot:

\begin{lstlisting}[language=tex,caption={Népszerű \LaTeX csomagok}]
  % A magyar nyelv támogatása
  \usepackage[magyar]{babel}

  % Dokuemntum margóinak beállítása (1 inch mindenhol)
  \usepackage[margin=1in]{geometry}

  % Matematikai szimbólumok, környezetek (több csomag egyszerre)
  \usepackage{amsmath,amssymb}

  % Grafikák beillesztéséhez
  \usepackage{graphicx}

  % Színes szöveg (68 alapszínnel)
  \usepackage[dvipsnames]{xcolor}

  % Hivasztások kezeléséhez (luatex módban)
  \usepackage[luatex]{hyperref}

  % Fej és lábléc kezeléséhez
  \usepackage{fancyhdr}
\end{lstlisting}

Mind az osztály, mind a csomagok betöltése esetén lehetőségünk van opcionális
argumentumok megadására is. Ezeket a kapcsos zárójelek közé írjuk. Osztály
esetén definiálhatjuk például az alapértelmezett betűméretet
(\inlinecode{[12pt]}),
a papírméretet
(\inlinecode{[a4paper]}),
vagy akár azt, hogy a dokumentum egy-, vagy kétoldalas legyen
(\inlinecode{[twoside]}).
Csomagok esetén pedig megadhatjuk például, hogy a csomag milyen nyelvet tölt be
(\inlinecode{\textbackslash usepackage[magyar]\{babel\}}),
a betölteni kívánt tulajdonságokat
(\inlinecode{\textbackslash usepackage[dvipsnames]\{xcolor\}}),
vagy akár azt, hogy a csomag milyen módban működjön
(\inlinecode{\textbackslash usepackage[luatex]\{hyperref\}}).

Ezek után a dokumentum többi tulajdonságait is beállíthatjuk. Megadhatjuk a
dokumentum címét, szerzőjét, dátumát.

\begin{lstlisting}[language=tex,caption={Dokumentum tulajdonságai}]
  \title{Cím}
  \author{Szerző neve}
  \date{Dátum}
\end{lstlisting}

Lehetőségünk van más beállítások módosítására is. Modosíthatjuk a dokumentum
oldalainak margóit, a fej és láblécet, a sorközt, és még sok más beállítást.

\begin{lstlisting}[language=tex,caption={Dokuemntum beállításai}]
  % Margók beállítása (\usepackage[margin=1in]{geometry}-vel ekvivalens)
  % \usepackage{geometry}
  \geometry{margin=1in}

  % Fejléc, lábléc, és stílus beállítása
  \usepackage{fancyhdr}
  \pagestyle{fancy}
  \lhead{Bal oldali fejléc} \rhead{Jobb oldali fejléc} \chead{Középső fejléc}
  \lfoot{Bal oldali lábléc} \rfoot{Jobb oldali lábléc} \cfoot{Középső lábléc}
\end{lstlisting}

Ha szeretnénk, akkor a dokumentum elején definiálhatunk saját parancsokat is.
Ezekről a későbbiekben lesz szó.

% ------------------------------------------------------------------------------
\subsection{A dokuemntum törzse}
% ------------------------------------------------------------------------------

% The document body
A dokumentum törzse a
\inlinecode{\textbackslash begin\{document\}}
parancs után kezdődik, és az
\inlinecode{\textbackslash end\{document\}}
parancs előtt ér véget. Ebben a részben definiálhatjuk a dokumentum tartalmát. A
törzsben használhatunk szöveget, matematikai képleteket, táblázatokat, képeket,
és még sok más elemet.

A több természetesen -- akár általunk definiált -- parancsokat is
alkalmazhatunk. A parancsokat a \texttt{\textbackslash} karakterrel kezdjük,
majd a parancs nevét írjuk le. A parancsok paramétereket is kaphatnak, amelyeket
kapcsos vagy szögletes zárójelek közé írunk attól függően, hogy a paraméter
kötelező vagy opcionális.

A \LaTeX{} dokumentumokban a szövegformázásra számos lehetőségünk van.
A leggyakrabban használtak:

\begin{minipage}[t]{.5\textwidth}
	\begin{lstlisting}[language=tex,caption={Szövegformázási parancsok}]
  % Betűtípus beállítása (mono, dőlt, félkövér)
  \texttt{mono} \textit{italic} \textbf{boldface}

  % Betűméret beállítása (apró, kicsi, nagy)
  {\tiny tiny} {\small small} {\large large}

  % Betűszín beállítása (piros, kék)
  {\color{red} red} {\color{blue} blue}

  % Aláhúzás, kiemelés
  \underline{underline} \emph{emphasis}
\end{lstlisting}
\end{minipage}\hfill%
\begin{minipage}[t]{.4\textwidth}
	\vspace{18pt}
	\texttt{mono} \textit{italic} \textbf{boldface}
	\\[18pt]
	{\tiny tiny} {\small small} {\large large}
	\\[18pt]
	{\color{red} red} {\color{blue} blue}
	\\[14pt]
	\underline{underline} \emph{emphasis}
\end{minipage}

A \LaTeX{} által biztosított környezetek segítségével különböző felsorolásokat
is létrehozhatunk. Egy ilyen környezet a
\inlinecode{\textbackslash begin\{<környezet neve>\}}
paranccsal kezdődik, és az
\inlinecode{\textbackslash end\{<környezet neve>\}}
paranccsal ér véget. A \texttt{enumerate} környezet sorszámozott, míg az
\texttt{itemize} környezet számozatlan felsorolást hoz létre.

\begin{minipage}[t]{.5\textwidth}
	\begin{lstlisting}[language=tex,caption={Felsorolások}]
  \begin{enumerate}
    \item Egy számozott elem
    \item Egy másik számozott elem
  \end{enumerate}

  \begin{itemize}
    \item Egy számozatlan elem
    \item Egy másik számozatlan elem
  \end{itemize}
\end{lstlisting}
\end{minipage}%
\begin{minipage}[t]{.5\textwidth}
	\vspace{12pt}
	\begin{enumerate}
		\item Egy számozott elem
		\item Egy másik számozott elem
	\end{enumerate}

	\begin{itemize}
		\item Egy számozatlan elem
		\item Egy másik számozatlan elem
	\end{itemize}
\end{minipage}

Matematikai képleteket az alábbi módon illeszthetünk be a dokumentumba:

\begin{minipage}[t]{.6\textwidth}
	\begin{lstlisting}[language=tex,caption={Matematikai képletek}]
  % Szövegbe ágyazott képlet
  A szög értéke $\alpha = 90^\circ$.

  % Különálló képlet
  \begin{equation}
    \alpha = 90^\circ
  \end{equation}

  % Különálló képlet, sorszámozás nélkül
  \begin{equation*}
    \alpha = 90^\circ
  \end{equation*}

  % Többsoros képlet, sorszámozással, =-nél igazítva
  % \usepackage{amsmath} (preambulumban)
  \begin{align}
    \alpha + \beta &= 90^\circ \\
    \gamma &= 90^\circ
  \end{align}

  % Többsoros képlet, sorszámozás és igazítás nélkül
  % \usepackage{amsmath} (preambulumban)
  \begin{gather*}
    \alpha + \beta = 90^\circ \\
    \gamma = 90^\circ
  \end{gather*}
\end{lstlisting}
\end{minipage}\hfill
\begin{minipage}[t]{.35\textwidth}
	\vspace{18pt}
	\phantom{alma} A szög értéke $\alpha = 90^\circ$.

	\vspace{15pt}
	\begin{equation}
		\alpha = 90^\circ
	\end{equation}

	\vspace{10pt}
	\begin{equation*}
		\alpha = 90^\circ
	\end{equation*}

	\vspace{10pt}
	\begin{align}
		\alpha + \beta & = 90^\circ \\
		\gamma         & = 90^\circ
	\end{align}

	\vspace{-10pt}
	\begin{gather*}
		\alpha + \beta = 90^\circ \\
		\gamma = 90^\circ
	\end{gather*}
\end{minipage}

Táblázatokat a \texttt{tabular} környezettel hozhatunk létre. A környezet
paramétereként megadhatjuk a táblázat oszlopainak számát, és az oszlopok
igazítását, valamint azt, hogy mely oszlopok között szeretnénk vízszintes
keretet. A sorokat az
\inlinecode{\textbackslash\textbackslash}
parancs választja el egymástól, míg az oszlopokat az
\inlinecode{\&}
karakter. A táblázat vízszintes kereteit a
\inlinecode{\textbackslash hline}
paranccsal húzhatjuk meg. A példában egy olyan táblázatot hozunk létre, amely
három oszlopot tartalmaz. Az első oszlop balra (\texttt{l}), a második középre
(\texttt{c}), a harmadik pedig jobbra (\texttt{r}) igazított. Az első oszlop
előtt, után, és a harmadik oszlop után vízszintes keretet húzunk (\texttt{|}).
A táblázat első sora felett, és az utolsó sora alatt szimpla, az első sora alatt
pedig dupla vonalas keretet húzunk.

\begin{minipage}[t]{.5\textwidth}
	\begin{lstlisting}[language=tex,caption={Táblázatok létrehozása}]
  \begin{tabular}{|l|c r|}
    \hline
    Bal & Közép & Jobb \\ \hline \hline
    123 & 23    & 3    \\
    4   & 567   & 6789 \\ \hline
  \end{tabular}
\end{lstlisting}
\end{minipage}\hfill%
\begin{minipage}[t]{.45\textwidth}
	\vspace{10pt}
	\begin{center}
		\begin{tabular}{|l|c r|}
			\hline
			Bal & Közép & Jobb \\ \hline \hline
			123 & 23    & 3    \\
			4   & 567   & 6789 \\ \hline
		\end{tabular}
	\end{center}
\end{minipage}

Ábrákat az \texttt{includegraphics} parancs segítségével illeszthetünk.
Opcionális paraméterként megadhatjuk a kép méreteit, tájolását, és még sok
mást.

\begin{minipage}[t]{.5\textwidth}
	\begin{lstlisting}[language=tex,caption={Ábrák beillesztése}]
  % \usepackage{graphicx} (preambulumban)
  \includegraphics[width=4cm]{./figures/bme_logo.pdf}
\end{lstlisting}
\end{minipage}\hfill%
\begin{minipage}[t]{.45\textwidth}
	\vspace{1pt}
	\begin{center}
		\includegraphics[width=4cm]{./figures/bme_logo.pdf}
	\end{center}
\end{minipage}

% ------------------------------------------------------------------------------
\subsection{A \LaTeX{} ökoszisztéma}
% ------------------------------------------------------------------------------

% CTAN
Előfordulhat, hogy dokumentumszerkesztés során olyan funkcionalitásra van
szükségünk, amelyet a \LaTeX{} alapértelmezetten nem biztosít. Ilyenkor a
\LaTeX{} -- ahogyan azt korábban is tárgyaltuk -- csomagok segítségével
bővíthetjük a rendszer funkcionalitását. A \textit{CTAN} (Comprenhensive \TeX{}
Archive Network) egy olyan központi tároló, ahol nemcsak maguk a \TeX{} és
\LaTeX{} csomagok, hanem azok dokumentációja is megtalálható. A \textit{CTAN}
a \TeX közösségben emiatt kulcsfontosságú szerepet tölt be, hiszen csomagok,
bővítmények, betűtípusok és egyéb tartalmak széles tárházát biztosítja a
felhasználók számára. Az archívum rendszeres látogatása mind kezdőknek, mind
haladóknak létfontosságú forrásként szolgál, hiszen nagymértékben megkönnyíti
a \LaTeX{}-kel kapcsolatos eszközök széles választékához való hozzáférést.
A hálózatot önkéntesek tartják fenn, tartalma pedig a világ számos pontján
lévő szerveren tükrözve van, ezzel garantálva a gyors és megbízható
elérhetőséget. \cite{ctan_homepage}

% ------------------------------------------------------------------------------
\section{Saját osztályok és csomagok}
% ------------------------------------------------------------------------------

% General advantages of using custom classes and packages
Az általam készített oktatási anyagokhoz saját osztályokat és csomagokat
készítettem. Nézzük meg először, hogy fejlesztés és megjelenés szempontjából
miért is volt előnyös ez a megoldás.

Saját készítésű forrásokat használva a dokumentumok felett teljes kontrollt
tudunk elérni. Mivel a parancsokat és környezeteket én definiáltam, ezért
a fejlesztés során egyrészt teljes szabadságot élveztem, másrészt pedig
forrás ismeretében a kimenet pontosan megjósolható volt. Ezzel a megközelítéssel
a dokumentumok fejlesztése és szerkesztése is sokkal egyszerűbbé vált.

Egy másik nagy előny, hogy ezáltal a dokumentumok megjelenése is egységes lett.
Gondoljunk csak bele, hogy ha a parancsokat minden egyes dokumentumban külön
definiáltam volna, akkor az nemcsak drasztikusan megnövelte volna a projekt
méretét, és csökkentette volna az átláthatóságát, de könnyen inkonzisztenciához
is vezethezett volna. Ha például a betűtípust szeretnénk megváltoztatni, akkor
azt nem csak a jelenleg szerkesztett dokumentumban, hanem az összes többiben
is meg kellett volna tennünk. Ha egy környezet viselkedését szerettük volna
megváltoztatni -- például újabb argumentumot adni hozzá -- akkor az összes
dokumentumban meg kellett volna keresni az adott környezet definícióját, és
meg kellett volna változtatni azt. Ezzel szemben a saját osztályok és csomagok
használatával a fejlesztés sokkal egyszerűbbé vált, hiszen amennyiben
változtatni szerettünk volna valamit, akkor csak az osztályt vagy a csomagot
kellett módosítani, és a változás az összes dokumentumra kiterjedt.

% ------------------------------------------------------------------------------
\subsection{Általános beállítások, egyszerű parancsok}
% ------------------------------------------------------------------------------

% General settings
A dokumentumok megjelenésének egységesítése érdekében különböző általános
beállítást vezettem be. Definiálva lett például a dokumentumok betűtípusa,
betűmérete. Különböző csomagokat hívtam meg, amelyeket az összes dokumentumban
használtam, így például a matematikai képletekhez szükséges csomagokat, a
grafikus elemeket kezelő csomagokat, és még sok mást.

% Simple commands
Létrehoztam alapvető parancsokat. Létrehoztam az általunk használatos
matematikai operátorokhoz tartozó parancsokat, definiáltam a gyakran használt
konstansokhoz tartozó szimbólumokat, és még sok egyszerű matematikai parancsot.
A skaláris szorzáshoz tartozó parancson keresztül megmutatom, hogy hogyan
lehet saját parancsokat definiálni. A célunk az, hogy egy olyan parancsot
hozzunk létre, amelynek két argumentuma van, és az alábbi megjelenést
eredményezi:
\[
	< a \, ; \, b >
	\text.
\]
Egy ilyen parancs létrehozása a \texttt{newcommand} parancs használatával
érhető el, melynek első paramétere a parancs neve, a következő -- opcionális --
paramétere a parancs argumentumainak száma, az utolsó pedig maga a parancs
kifejtése, ahol az argumentumokat \texttt{\#1}, \texttt{\#2}, \ldots,
\texttt{\#n} formában érhetjük el. Fontos, hogy a parancs neve csak betűket
tartalmazhat, valamint maximálisan kilenc paramétert várhat. A megvalósítás
a következőképpen néz ki:

\begin{lstlisting}[language=tex,caption={Skaláris szorzás parancs definíciója]}]
  % \newcommand{<parancs neve>}[<argumentumok száma>]{<parancs definíciója>}
  \newcommand{\scalar}[2]{< #1 \, ; \, #2 >}
\end{lstlisting}

A parancs neve \texttt{scalar}, az argumentumok száma pedig kettő. Amennyiben
a parancsot használni szeretnénk, akkor azt a beépített parancsokhoz hasonlóan
tehetjük meg. A parancs neve után a parancs argumentumait kapcsos zárójelek
közé írjuk, vesszővel elválasztva. A parancs használata a következőképpen
néz ki:

\begin{lstlisting}[language=tex,caption={Skaláris szorzás parancs használata}]
  \scalar{a}{b}
\end{lstlisting}

Most nézzünk meg egy bonyolultabb példát. Definiáljunk egy megjegyzés
környezetet, célunk az alábbi kinézet elérése:

\begin{note}[Opcionális cím]
	\noindent Ez egy megjegyzés.
\end{note}
\begin{note}
	\noindent Ez egy másik megjegyzés, cím nélkül.
\end{note}

A kinézet eléréséhez az \texttt{mdframed} csomagot használtam, amely
segítségével definiáltam egy új, saját környezetet:

\begin{lstlisting}[language=tex,caption={Saát megjegyzés környezet}]
  \usepackage{mdframed}
  \newmdenv[
    backgroundcolor=yellow-base!5,%  Háttérszín (kevert szín)
    linecolor=blue-base,%            Keret színe
    linewidth=1mm,%                  Keret vastagsága
    topline=false,%                  Felső vonal nincs
    rightline=false,%                Alsó vonal nincs
    bottomline=false,%               Jobb oldali vonal nincs
    skipabove=0.25em,%               Felső margó
    skipbelow=0.25em,%               Alsó margó
  ]{myframe}
\end{lstlisting}

Ezután következett a környezet definiálása. A \texttt{newenvironment} parancs
első paramétere a környezet neve, a következő -- opcionális -- paramétere a
környezet argumentumainak száma, mely után megadhatjuk a környezet opcionális
paramétereinek alapértelmezett értékeit. Az utolsó két paraméter pedig a
környezet kezdő és záró parancsainak definíciója. Mivel szeretnénk, hogy a
megjegyzések sorszámozva is legyenek, ezért egy számlálot is létre kellett
hoznunk. A megvalósítás a következőképpen néz ki:

\begin{lstlisting}[language=tex,caption={Megjegyzés környezet}]
  % \newenvironment
  %   {<környezet neve>}
  %   [<argumentumok száma>]
  %   [<opcionális paraméterek>]
  %   {<kezdő parancs>}
  %   {<záró parancs>}
  \newcounter{note}
  \makeatletter
  \newenvironment{note}[1][\@nil]{%
    \refstepcounter{note}%
    \begin{myframe}\textbf{Megjegyzés \thenote.}%
      \def\tmp{#1}%
      \ifx\tmp\@nnil\else%
        \hspace{.5em}[\;#1\;]
      \fi\par%
  }{\end{myframe}}
  \makeatother
\end{lstlisting}

Először is létrehoztam egy számlálót, amely a megjegyzések sorszámát tartja
számon. Ezután egy eddig nem látott paranccsal találkozhatunk: a környezet
létrehozása előtt a
\inlinecode{\textbackslash makeatletter},
utána pedig a
\inlinecode{\textbackslash makeatother}
parancsot adtuk ki.

Hogy mit is csinál ez? A fejlesztők gyakran \texttt{@}
karaktereket raknak az olyan parancsok, környezetek nevébe, amelyeket nem
közvetlen használatra vannak kitalálva. A \LaTeX{} alapvetően ezt a karaktert
nem betűként kezeli, ami azt jelenti, hogy az ilyen makrók alapértelmezett
üzemmódban nem használhatóak. A
\inlinecode{\textbackslash makeatletter}
paranccsal a \LaTeX{} fordító számára azt jelezzük, hogy az \texttt{@} karaktert
betűként kezelje. Ezáltal az ilyen karaktereket tartalmazó parancsokat is
elérhetjük. A környezet létrehozása után érdemes a
\inlinecode{\textbackslash makeatother}
parancsot kiadni, hogy a \LaTeX{} fordító ismét az alapértelmezett üzemmódba
kapcsoljon vissza.

Térjünk vissza a környezet létrehozásához. A környezetnek egy opcionális
paramétert adtunk meg, amely a megjegyzés címét tartalmazza. Ennek az
alapértelmezett értéke
\inlinecode{\textbackslash @nil}.
A környezet kezdő parancsában először megnöveltük a számláló értéket.
Utána a korábban definiált keret környezetet hívtuk meg. Ebben először
kiírtuk a megjegyzés sorszámát. Utána egy ideiglenes változóba elmentjük
a kapott argumentumot, amely -- akármennyire is redundánsnak tűnik --  a későbbi
elágazás miatt egy nem elhagyható lépés. Mivel amennyiben nincs cím megadva,
akkor a címet körül vevő kapcsos zárójeleket sem szeretnénk megjeleníteni,
ezért egy elágazással ellenőrizzük, hogy a kapott cím \texttt{\textbackslash
	@nil}-e. Amennyiben nem, akkor 0,5 em hely kihagyása után
kiírjuk a címet. A környezet záró parancsában pedig a korábban definiált
keret környezet záró parancsát hívtuk meg.

% ------------------------------------------------------------------------------
\subsection{A feladat környezet}
% ------------------------------------------------------------------------------

% The exercise environment
A legbonyolultabb kihívás a feladat (\texttt{exercise}) környezet létrehozása
volt. Az ezzel kapcsolatos elvárások a következőek voltak:
\begin{itemize}
	\item Legyen egy opcionális, és egy kötelező paramétere. Utóbbi a feladat
	      címe. Előbbi pedig egy olyan hosszmérték amennyit ki szeretnénk hagyni,
	      amennyiben a megoldás nincs kiiratva. Ennek alapértelmezett értéke
	      5 cm.

	      % \item A feladatok számozva legyenek.

	\item Legyen egy külön csomagban definiálva, mely csomagnak opcionálisan
	      paramétereket adhatunk meg.
	      \begin{itemize}
		      \item Amennyiben a \texttt{print-solutions} paramétert megadjuk,
		            akkor a feladatok megoldásai is megjelennek.

		      \item Ha a \texttt{keep-space} paramétert adjuk meg, akkor
		            a feladatok után az általunk megadott hosszúságú, vagy
		            alapértelmezetten 5 cm hosszú üres hely marad.

		      \item Ez a két opció együtt nem használható. Amennyiben mindkettőt
		            megadjuk, akkor a \texttt{keep-space} paramétert figyelmen kívül
		            hagyjuk, a felhasználót pedig egy figyelmeztetéssel
		            tájékoztatjuk.

		      \item Ha ezek közül egyiket sem adjuk meg, akkor sem üres hely,
		            sem megoldás nem jelenik meg.
	      \end{itemize}
\end{itemize}

A fájl neve legyen \texttt{math-exercise.sty}. Vegyük észre, hogy nem
\texttt{.tex}, hanem \texttt{.sty} kiterjesztést használtunk. Ennek két előnye
van. Egyrészt a \LaTeX{} fordító számára ez egy csomagot jelent, és nem egy
dokumentumot, így a preambulumban a \texttt{usepackage} paranccsal tudjuk
meghívni. Másrészt pedig mivel ez a fájltípus kifejezetten a \LaTeX{} csomagok
számára lett kitalálva, ezért például használhatjuk az \texttt{@} karaktert
a parancsok nevében, anélkül, hogy előrre a \texttt{makeatletter} parancsot
kellene kiadnunk.

Először meg kell adnunk, hogy milyen \TeX{} formátumot használunk, valamint
a csomag nevét, esetleg verzióját:

\begin{lstlisting}[language=tex,caption={Csomag információk}]
  \NeedsTeXFormat{LaTeX2e}
  \ProvidesPackage{math-exercise}
\end{lstlisting}

Ezután meghívhatunk különböző csomagokat, amelyekre szükségünk van. Figyeljük
meg, hogy ezekez nem a \texttt{usepackage}, hanem a \texttt{RequirePackage}
parancca hívjuk meg. Ennek leginkább formális jelentősége van, hiszen a két
parancs között nincs funkcionális különbség. Az egyetlen differencia, hogy
a \texttt{usepackage} parancs nem használható a dokumentum osztályának megadása
előtt.

\begin{lstlisting}[language=tex,caption={Csomagok meghívása}]
  \RequirePackage{amsmath}                   % Matematika környezetek
  \RequirePackage{xargs,xstring}             % Argumentumok kezelése
  \RequirePackage{tikz}                      % Ábrák
  \usetikzlibrary{patterns,patterns.meta}    % Mintázatok
  \RequirePackage[many]{tcolorbox}           % Színes dobozok
  \tcbuselibrary{breakable}                  % A dobozok több oldalasak is lehetnek
\end{lstlisting}

Ezután a csomag argumentumait fogjuk feldolgozni. Kettő fajta opciót adhatunk
meg, ezeknek létrehozunk egy-egy boolean tárolót a \texttt{newif} parancs
segítségével. Ellenőrizzük, hogy a két opció közül csak az egyiket adjuk meg,
ha mindkettőt megadjuk, akkor figyelmeztetjük a felhasználót, és a
\texttt{print-solutions} opciót választjuk.

\begin{lstlisting}[language=tex,caption={Opciók feldolgozása}]
  % Tárolók létrehozása
  \newif\ifprint@@sols
  \newif\ifkeep@@space

  % Opciók feldolgozása
  \DeclareOption{print-solutions}{\print@@solstrue}
  \DeclareOption{keep-space}{%
    % Ha a 'print-solutions' opciót is megadtuk, akkor figyelmeztetünk
    \ifprint@@sols%
      \PackageWarning{math-exercise}%
      {'keep-space' ignored, because it cannot be used together with 'print-solutions'}%
    % Ha nem, akkor beállítjuk a 'keep-space' opciót
    \else%
      \keep@@spacetrue%
    \fi%
  }

  % Opciók feldolgozásának vége
  \ProcessOptions\relax
\end{lstlisting}

Új parancsot itt is a \texttt{newcommand}, új környezetet pedig a
\texttt{newenvironment} paranccsal hozhatunk létre. A feladat környezet
létrehozásához az \texttt{exercise} elnevezést használtam. Létrehoztam továbbá
egy \texttt{exsol} parancsot, amely a feladat megoldását tartalmazza. Ez
csak a feladat környezetben definiálható, és csak akkor fog megjelenni,
amennyiben a \texttt{print-solutions} opciót megadtuk.

Először a paraméterekhez tartozó színeket és mintázatokat definiáltam.
Amennyiben a \texttt{keep-space} opció aktív, akkor a kihagyott üres hely
ponthálós hátteret kap. Egyébként pedig üresen hagyjuk a hátteret.

\begin{lstlisting}[language=tex,caption={Konstansok definiálása}]
  \def\@@cb{yellow-base!5}%
  \def\@@cf{blue-base}%
  \def\@@ct{white}

  % Üres hely mintázata, amennyiben a 'keep-space' opció aktív
  \ifkeep@@space%
    \def\@@ul{%
      \begin{tcbclipinterior}
        \fill[
        pattern={%
          Dots[distance=5mm, angle=90],%
        },%
        pattern color=gray,
        shift={(interior.north west)},
        ] (interior.south west) rectangle (segmentation.east);
      \end{tcbclipinterior}
    }
  \else%
    \def\@@ul{}
  \fi%
\end{lstlisting}

Most már minden adott volt, hogy definiáljuk a feladat környezetet. Létrehoztam
egy olyan tárolót, amely jelzi, hogy a \texttt{exsol} parancs használható-e.
Ez azért fontos, mert a \texttt{exsol} parancs csak a \texttt{exercise}
környezetben használható. Ezen kívül egy számolóra is szükség van, hiszen a
megjegyzésekhez hasonlóan a feladatokat is szeretnénk sorszámozni.

\begin{lstlisting}[language=tex,caption={Az \texttt{exercise} környezet}]
  \newif\ifinside@@exercise%              % 'exsol' parancs használható-e?
  \newcounter{exercise}%                  % Feladatok számozása

  \newenvironment{exercise}[1]{%
    \inside@@exercisetrue%                % 'exsol' parancs használható
    \refstepcounter{exercise}%            % számláló inkrementálása
    \begin{tcolorbox}[%                   % Feladat doboz eleje
      colback=\@@cb,%                     % Háttérszín
      colframe=\@@cf,%                    % Keret színe
      coltitle=\@@ct,%                    % Cím színe
      breakable,%                         % Több oldalas
      title={\#\theexercise . #1},%       % Cím
      enhanced,                           % Bővített módban
      underlay={\@@ul},                   % Üres hely mintázata
    ]%
  }%
	{%
    \inside@@exercisefalse%               % 'exsol' parancs nem használható
    \end{tcolorbox}%                      % Feladat doboz vége
  }%
\end{lstlisting}

Már csak a \texttt{exsol} parancs definiálása van hátra. A parancs első
paramétere opcionális, a kihagyott távolságot tartalmazza. Az ezt követő
paraméter pedig maga a megoldás.

\begin{lstlisting}[language=tex,caption={Az \texttt{exsol} parancs}]
  \newcommandx{\exsol}[2][1=5cm]{%
    \ifinside@@exercise%                  % Ha az 'exercise' környezetben vagyunk
      \ifprint@@sols%                     % Ha a 'print-solutions' opció aktív
        \tcblower%                        % Elválasztó vonal
        #2%                               % Megoldás kiiírása
      \fi%
      \ifkeep@@space%                     % Ha a 'keep-space' opció aktív
        \IfInteger{#1}{%                  % Ha a paraméter egész szám
          \ifnum#1>0%                     % Ha a paraméter pozitív
            \tcblower%                    % Elválasztó vonal
            \vspace{#1}%                  % Üres hely
          \fi
        }{%
          \tcblower%                      % Elválasztó vonal
          \vspace*{#1}%                   % Üres hely
        }%
      \fi%
    \else%                                % Ha nem az 'exercise' környezetben vagyunk
      \PackageError{math-exercise}%
      {'exsol' can only be used in the exercise environment}%
    \fi%
  }
\end{lstlisting}

\begin{minipage}[t]{.33\textwidth}
	\begin{lstlisting}[language=tex,caption={Csomag használata}]
  \usepackage[
    print-solutions
  ]{math-exercise}





  \usepackage[
    keep-space
  ]{math-exercise}






  \usepackage{exercise}


  \end{lstlisting}
\end{minipage}\hfill
\begin{minipage}[t]{.63\textwidth}
	\begin{exercise}{Megoldással}
		Mennyi $2+2$?

		\exsol{%
			\vspace{-8mm}
			\[
				2 + 2 = 4
			\]
			\vspace{-10mm}
		}
	\end{exercise}
	\vspace{-1.5mm}

	\makeatletter
	\print@@solsfalse
	\keep@@spacetrue
	\begin{exercise}{Helykihagyással}
		Mennyi $2+2$?

		\exsol[11mm]{}
	\end{exercise}
	\vspace{-1.5mm}

	\keep@@spacefalse
	\begin{exercise}{Megoldás és helykihagyás nélkül}
		Mennyi $2+2$?
	\end{exercise}
	\makeatother
\end{minipage}



% ------------------------------------------------------------------------------
\subsection{Az fájltípusok és a hozzájuk tartozó környezetek}
% ------------------------------------------------------------------------------
