\newcommand{\tss}{\textsuperscript}
%-------------------------------------------------------------------------------
\chapter*{\jelolesek}\addcontentsline{toc}{chapter}{\jelolesek}
%-------------------------------------------------------------------------------

A táblázatban a többször előforduló jelölések magyar és angol nyelvű elnevezése,
valamint a fizikai mennyiségek esetén annak mértékegysége található. Az egyes
mennyiségek jelölése – ahol lehetséges – megegyezik hazai és a nemzetközi
szakirodalomban elfogadott jelölésekkel. A ritkán alkalmazott jelölések
magyarázata első előfordulási helyüknél található.

%~~~~~~~~~~~~~~~~~~~~~~~~~~~~~~~~~~~~~~~~~~~~~~~~~~~~~~~~~~~~~~~~~~~~~~~~~~~~~~~~~~~~~
% A táblázatokat ABC rendben kell feltölteni, először mindig a kisbetűvel
% kezdve. Ha egyazon betűjelnek több értelmezése is van, akkor mindegyiket kü-
% lön sorban kell feltüntetni. Konstansok esetén az értéket is a táblázatba
% kell írni.
% Dimenzió nélküli mennyiségek mértékegysége 1 és nem: – !
% A jelölésjegyzékben csak SI vagy SI-n kívüli engedélyezett mértékegységeket
% szabad feltüntetni. Egy dokumentumon belül az SI és pl. az angolszász
% mértékrendszer nem keverhető!
%~~~~~~~~~~~~~~~~~~~~~~~~~~~~~~~~~~~~~~~~~~~~~~~~~~~~~~~~~~~~~~~~~~~~~~~~~~~~~~~~~~~~~

%~~~~~~~~~~~~~~~~~~~~~~~~~~~~~~~~~~~~~~~~~~~~~~~~~~~~~~~~~~~~~~~~~~~~~~~~~~~~~~~~~~~~~
% A Jelölés oszlop alapvetően kurzív betűváltozattal szedendő, a Mértékegység
% oszlopot álló betűkkel kell szedni. Felső indexhez használható a \tss{}
% parancs.
%~~~~~~~~~~~~~~~~~~~~~~~~~~~~~~~~~~~~~~~~~~~~~~~~~~~~~~~~~~~~~~~~~~~~~~~~~~~~~~~~~~~~~

\def\arraystretch{1.5}%  vertical cell padding

\subsubsection*{Latin betűk}
\begin{center}
  \begin{tabular}{lp{10cm}l}
    \hline
    Jelölés & Megnevezés, megjegyzés, érték & Mértékegység   \\
    \hline
    $g$     & gravitációs gyorsulás (9.81)  & m/s\tss{2}     \\
    $p$     & nyomás                        & bar            \\
    $s$     & fajlagos entrópia             & J/(kg$\cdot$K) \\
    \hline
  \end{tabular}
\end{center}



\subsubsection*{Görög betűk}
\begin{center}
  \begin{tabular}{lp{10cm}l}
    \hline
    Jelölés & Megnevezés, megjegyzés, érték & Mértékegység \\
    \hline
    $\eta$  & hatásfok                      & 1            \\
    $\rho$  & sűrűség                       & kg/m\tss{3}  \\
    \hline
  \end{tabular}
\end{center}



\subsubsection*{Indexek, kitevők}
\begin{center}
  \begin{tabular}{lp{12.8cm}}
    \hline
    Jelölés & Megnevezés, értelmezés           \\
    \hline
    $i$     & általános futóindex (egész szám) \\
    nom     & névleges (nominális) érték       \\
    opt     & legkedvezőbb (optimális) érték   \\
    \hline
  \end{tabular}
\end{center}


\def\arraystretch{1}%  vertical cell padding
