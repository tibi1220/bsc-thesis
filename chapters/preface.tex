%-------------------------------------------------------------------------------
\chapter*{\eloszo}\addcontentsline{toc}{chapter}{\eloszo}
%-------------------------------------------------------------------------------

A COVID-19 világjárvány oktatásra gyakorolt tartós hatása várhatóan még hosszú
évekig érezhető lesz. Azok a hallgatók, akik 2022-ben kezdték meg egyetemi
tanulmányaikat azzal a kihívással szembesültek, hogy középiskolai tanulmányaikat
a járvány tetőpontján fejezték be, amely egy döntő fontosságú időszak
tanulmányaik szempontjából. A korábbi évekhez képest a Matematika G1
tantárgy vizsgáinak átmeneti aránya jelentősen csökkent, annak ellenére,
hogy rendelkezésre álltak különböző segédeszközök, például interaktív online
felületek, oktatófilmek, átfogó jegyzetek és bonyolult számítási feladatok.
Sajnálatos módon sok tanuló küzdött a felzárkózással és az önálló haladással.

A szakdolgozatom célja, hogy bemutassam, hogy az általam készített jegyzetekkel
hogyan lehet segíteni a hallgatókat a tanulásban. Bemutatom, hogy milyen
technológiákat használtam a jegyzetek elkészítéséhez, milyen filozófiák
által vezérelve készítettem őket, és hogyan lehet ezeket a jegyzeteket
az oktatásban felhasználni.


\begin{center}
	$\thicksim \; \thicksim \; \thicksim$
\end{center}

\subsubsection*{Köszönetnyilvánítás}

\begin{center}
	\Huge
	TODO: THANKS GOES HERE
\end{center}

\vspace{0.5cm}

\begin{flushleft}
	{Budapest, \today}
\end{flushleft}

\begin{flushright}
	\emph{\authorName}
\end{flushright}

\vfill
